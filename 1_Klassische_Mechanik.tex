\section{Klassische Mechanik}
\subsection{Abriss der Newtonschen Mechanik}
Problemstellung der Mechanik:
Orte ri und Geschwindigkeiten vi zur Zeit t0 gegeben für ein System von Massepunkten 1 <= i <= N. Und es wirken äußere Kräfte F und Kräfte zwischen den Teilchen i und j Fij.
Wie lauten die \underline{kinematischen Größen} ri(t), vi(t) = ri.(t) für beliebige Zeiten t danach?
Die kinematischen Größen ri(t) und ri.(t), ri..(t) werden als Lösung ordentlicher Differentialgleichungen gefunden. Das sind die \underline{Bewegungsgleichungen}.
Neben den kinematische Größen gibt des die wichtigen Begriffe \underline{Kraft, Masse und Impuls}.

\subsubsection{Kraft}
\begin{itemize}
  \item vektorielle Größe F
  \item Ursache der Bewegung, bewirkt Änderung des Bewegungszustandes.\\
        ~> kräftefrei: Bewegungszustand unverändert\\
        -> \underline{Newtonsche Gesetze}
\end{itemize}

\subsubsection{Newtonsche Gesetze}
* \underline{lex prima (Galileische trägheitsgesetz)}
Es gibt \underline{Inertialsysteme}, in denen ein kräftefreier Körper (Massepunkt) ruht oder sich geradlinig, gleichförmig bewegt.

\underline{Definition:} Jeder Messapunkt setzt der Einwirkung von Kräften einen Trägheitswiderstand entgegen = \underline{Masse} (träge Masse)

Damit \underline{Definition Impuls}:
p = m*v

* \underline{lex secundo (Newtonsches Bewegungsgesetz)}
p. = F
v. = d/dt v = a -> F=m*a

* \underline{lex tertia (actio = reactio)}
Fij = -Fji

=> Defintion der trägen Masse unabhängig von der Kraft
\textbf{BILD}

=> Verhältnix der Geschwindigkeiten unabhängig von |F| -> Definition träge Masse.

\underline{Beispiele für Kräfte}
\begin{itemize}
  \item Gravitationskraft (schwere Masse = träge Masse)\\
        \textbf{BILD}\\
        Speziell auf der Erde:\\ \textbf{BILD}
  \item Coulombkraft zwischen elektrischen Ladungen Q1, Q2\\
        F = 1/(4*pi*eps0) * Q1Q2/r^2 * ^r\\
        1 / (4 * pi * eps0) = const
  \item Lorentzkraft: F 0 e * (E + v x B) \textbf{Anmerkungen BILD}
\end{itemize}