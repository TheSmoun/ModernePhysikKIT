\chapter{Klassische Mechanik}
\section{Abriss der Newtonschen Mechanik}
Problemstellung der Mechanik:
Orte ri und Geschwindigkeiten vi zur Zeit t0 gegeben f\"ur ein System von Massepunkten 1 <= i <= N. Und es wirken \"außere Kr\"afte F und Kr\"afte zwischen den Teilchen i und j Fij.
Wie lauten die \underline{kinematischen Gr\"oßen} ri(t), vi(t) = ri.(t) f\"ur beliebige Zeiten t danach?
Die kinematischen Gr\"oßen ri(t) und ri.(t), ri..(t) werden als L\"osung ordentlicher Differentialgleichungen gefunden. Das sind die \underline{Bewegungsgleichungen}.
Neben den kinematische Gr\"oßen gibt des die wichtigen Begriffe \underline{Kraft, Masse und Impuls}.

\subsection{Kraft}
\begin{itemize}
  \item vektorielle Gr\"oße F
  \item Ursache der Bewegung, bewirkt Änderung des Bewegungszustandes.\\
        ~> kr\"aftefrei: Bewegungszustand unver\"andert\\
        -> \underline{Newtonsche Gesetze}
\end{itemize}

\subsection{Newtonsche Gesetze}
* \underline{lex prima (Galileische tr\"agheitsgesetz)}
Es gibt \underline{Inertialsysteme}, in denen ein kr\"aftefreier K\"orper (Massepunkt) ruht oder sich geradlinig, gleichf\"ormig bewegt.

\underline{Definition:} Jeder Messapunkt setzt der Einwirkung von Kr\"aften einen Tr\"agheitswiderstand entgegen = \underline{Masse} (tr\"age Masse)

Damit \underline{Definition Impuls}:
p = m*v

* \underline{lex secundo (Newtonsches Bewegungsgesetz)}
p. = F
v. = d/dt v = a -> F=m*a

* \underline{lex tertia (actio = reactio)}
Fij = -Fji

=> Definition der tr\"agen Masse unabh\"angig von der Kraft
\textbf{BILD}

=> Verh\"altnis der Geschwindigkeiten unabh\"angig von |F| -> Definition tr\"age Masse.

\underline{Beispiele f\"ur Kr\"afte}
\begin{itemize}
  \item Gravitationskraft (schwere Masse = tr\"age Masse)\\
        \textbf{BILD}\\
        Speziell auf der Erde:\\ \textbf{BILD}
  \item Coulombkraft zwischen elektrischen Ladungen $Q_{1}$ und $Q_{2}$\\
        \(\vec{F} = \frac{1}{4*\pi*\epsilon_{0}} * Q_{1}Q_{2}/r^{2} * \hat{r}\\
        \frac{1}{4*\pi*\epsilon_{0}} = const\)
  \item Lorentzkraft: $\vec{F} = e * (\vec{E} + \vec{v} \times \vec{B})$ \textbf{Anmerkungen BILD}
\end{itemize}