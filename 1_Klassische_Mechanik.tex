\section{Klassische Mechanik}
\subsection{Abriss der Newtonschen Mechanik}
Problemstellung der Mechanik:
Orte ri und Geschwindigkeiten vi zur Zeit t0 gegeben für ein System von Massepunkten 1 <= i <= N. Und es wirken äußere Kräfte F und Kräfte zwischen den Teilchen i und j Fij.
Wie lauten die \underline{kinematischen Größen} ri(t), vi(t) = ri.(t) für beliebige Zeiten t danach?
Die kinematischen Größen ri(t) und ri.(t), ri..(t) werden als Lösung ordentlicher Differentialgleichungen gefunden. Das sind die \underline{Bewegungsgleichungen}.