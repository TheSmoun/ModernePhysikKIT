Die Moderne Physik wird meist im Gegensatz zur "klassischen Physik" gesehen.\\
Bis zum Anfang des 20. Jahrhunderts (1. Viertel) gab es die klassische Physik mit Newtons Mechanik und Maxwells Elektrodynamik. Das Paradigma der klassischen Physik ist: "alles ist im Prinzip berechenbar".\\
$\rightarrow$ Zeitentwicklung von Systemen von Massepunkten, solange die Anfangsbedingung bekannt ist
$x_{i}(t_{0})$, $\dot{x_{i}}(t_{0})$\\
\\
\underline{Aber:} Die Experimente zeigten immer mehr Widerspr\"uche.
\begin{itemize}
  \item Michelson-Morley: Es gibt "keinen \"Ather"\\
        Die Lichtgeschwindigkeit (\textbf{c}) ist konstant.\\
        $\Rightarrow$ \underline{spezielle Relativit\"atstheorie (SRT)}
  \item Diskrete Emissionsspektren (Spektrallinien)
  \item Welleneigenschaft von Teilchen \textbf{BILD}
  \item Teilcheneigenschaft von Lichtwellen
  \item Schwarzk\"orperspektrum\\
        Identische Teilchen
        $\Rightarrow$ Quantenphysik
\end{itemize}

\section{Themen\"uberblick}
\subsection{Klassische Mechanik}
Newtonsche Mechanik\\
$\rightarrow$ Entwicklung der Formalen Mechanik (Lagrange, Hamilton, Jacobi)\\
$\rightarrow$ erlaubt eine theoretische Diskussion der Mechanik\\
Symmetrien und weitere wichtige Konzepte, die in der \underline{Quantenmechanik (QM)} gebraucht werden\\
"Hamiltonoperator", kanonisch konjungierte Variable $\rightarrow$ Symmetrien von Erhaltungss\"atzen

\subsection{Relativit\"at}
Symmetrie von Raum und Zeit\\
$\rightarrow$ SRT (etwas losgel\"ost von Mechanik)\\
Formale Entwicklung der Theorie\\
$\rightarrow$ radikale Konsequenzen
(evtl. etwas allgemeine Relativit\"atstheorie)

\subsection{Quantenmechanik (QM)}
	- etwas Historie\\
	- einfache, 1-dimensionale Probleme\\
	$\rightarrow$ Schr\"odingergleichung, Ortsdarstellung\\
	$\rightarrow$ "Wellenmechanik"\\
	- Postulate der QM\\
	- Symmetrie und Erhaltungss\"atze (Drehimpuls, Spin)\\
	- Wasserstoffatom\\
	$\rightarrow$ Periodensystem der Elemente (PSE)\\
	- Identische Teilchen (Bosonen und Fermionen)\\
	- Beispiele