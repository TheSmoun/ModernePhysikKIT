\section{Übersicht}
Gegensatz zur "klassischen Physik"
Anfang des 20. Jahrhunderts (1. Viertel)
Bis dahin gab es die klassische Physik mit Newtons Mechanik und Maxwells Elektrodynamik.
Paradigma: alles ist im Prinzip berechenbar.
-> Zeitentwicklung von Systemen von Massepunkten, solange die Anfangsbedingung bekannt ist
xi(t0); xi.(t0)

\underline{Aber:} Experimente zeigen immer mehr Widersprüche.
\begin{itemize}
  \item Michelson-Morley: Es gibt "keinen Äther"\\
        Die Lichtgeschwindigkeit (\textbf{c}) ist konstant.\\
        => \underline{spezielle Relativitätstheorie (SRT)}
  \item Diskrete Emmisionsspektren (Spektrallinien)
  \item Welleneigenschaft von Teilchen \textbf{BILD}
  \item Teilcheneigenschaft von Lichtwellen
  \item Schwarzkörperspektrum\\
        Identische Teilchen
        => Quantenphysik
\end{itemize}

\section{Themenüberblick}
\subsection{Klassische Mechanik}
- Newtonsche Mechanik
-> Entwicklung der Formalen Mechanik (Lagrange, Hamilton, Jacobi)
-> erlaubt eine theoretische Diskussion der Mechanik

- Symmetrien und weitere wichtige Konzepte, die in der \underline{Quantenmechanik (QM)} gebraucht werden
- "Hamiltonoperator", kanonisch konjungierte Variable
-> Symmetrien von Erhaltungssätzen

\subsection{Relativität}
- Symmetrie von Raum und Zeit
-> SRT (etwas losgelöst von Mechanik)
Formale Entwicklung der Theorie
-> radikale Konsequenzen
(evtl. etwas allgemeine Relativitätstheorie)

\subsection{Quantenmechanik (QM)}
- etwas Historie
- einfache, 1-dimensionale Probleme
-> Schrödingergleichung, Ortsdarstellung
-> "Wellenmechanik"
- Postulate der QM
- Symmetrie und Erhaltungssätze (Drehimpuls, Spin)
- Wasserstoffatom
-> Periodensystem der Elemente (PSE)
- Identische Teilchen (Bosonen und Fermionen)
- Beispiele