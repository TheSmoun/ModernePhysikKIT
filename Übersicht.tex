Gegensatz zur "klassischen Physik"
Anfang des 20. Jahrhunderts (1. Viertel)
Bis dahin gab es die klassische Physik mit Newtons Mechanik und Maxwells Elektrodynamik.
Paradigma: alles ist im Prinzip berechenbar.
-> Zeitentwicklung von Systemen von Massepunkten, solange die Anfangsbedingung bekannt ist
$x_{i}(t_{0})$, $\dot{x_{i}}(t_{0})$

\underline{Aber:} Experimente zeigen immer mehr Widerspr\"uche.
\begin{itemize}
  \item Michelson-Morley: Es gibt "keinen \"Ather"\\
        Die Lichtgeschwindigkeit (\textbf{c}) ist konstant.\\
        => \underline{spezielle Relativit\"atstheorie (SRT)}
  \item Diskrete Emissionsspektren (Spektrallinien)
  \item Welleneigenschaft von Teilchen \textbf{BILD}
  \item Teilcheneigenschaft von Lichtwellen
  \item Schwarzk\"orperspektrum\\
        Identische Teilchen
        => Quantenphysik
\end{itemize}

\section{Themen\"uberblick}
\subsection{Klassische Mechanik}
- Newtonsche Mechanik
-> Entwicklung der Formalen Mechanik (Lagrange, Hamilton, Jacobi)
-> erlaubt eine theoretische Diskussion der Mechanik

- Symmetrien und weitere wichtige Konzepte, die in der \underline{Quantenmechanik (QM)} gebraucht werden
- "Hamiltonoperator", kanonisch konjungierte Variable
-> Symmetrien von Erhaltungss\"atzen

\subsection{Relativit\"at}
- Symmetrie von Raum und Zeit
-> SRT (etwas losgel\"ost von Mechanik)
Formale Entwicklung der Theorie
-> radikale Konsequenzen
(evtl. etwas allgemeine Relativit\"atstheorie)

\subsection{Quantenmechanik (QM)}
- etwas Historie
- einfache, 1-dimensionale Probleme
-> Schr\"odingergleichung, Ortsdarstellung
-> "Wellenmechanik"
- Postulate der QM
- Symmetrie und Erhaltungss\"atze (Drehimpuls, Spin)
- Wasserstoffatom
-> Periodensystem der Elemente (PSE)
- Identische Teilchen (Bosonen und Fermionen)
- Beispiele